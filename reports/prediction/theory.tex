\section{Theory}
\subsection{Normalization}
\subsubsection{Pixel values}
To reduce the risk of exploding and/or vanishing gradients, pixel values are 
commonly normalized to $[0, 1]$ or to $[-1, 1]$ (with zero-mean). 
Here we have opted for a min-max scaling to the $[0, 1]$ interval.
For a set of images the minimum and maximum pixel values across all images
in the set are the ones used in calculations. This preserves differences in
intensities between images in the set, and also the shape of the intensity 
distribution. Min-max scaling is calculated as
\begin{equation}
    \text{scaled image} = \f{\text{image} - \mu_{image}}{I_{max} - I_{min}},
\end{equation}
where $I_{max}$ and $I_{min}$ refer to the maximum and minimum pixel intensity,
and $\mu_{image}$ is the mean pixel intensity for the set of images.

\subsection{Distances}
The data provides position of origin in units of pixels, such that coordinates
$x,y \in [0,16]$. We normalize these to the interval $[0,1]$ by dividing all
distances by 16.
